\documentclass[11pt]{article}

% --- Packages ---
\usepackage[utf8]{inputenc}
\usepackage{amsmath, amssymb, amsthm, amsfonts}
\usepackage{mathtools}       % For extra math tools
\usepackage{bm}              % Bold math symbols
\usepackage{enumitem}        % Better control over lists
\usepackage{geometry}        % Better margins
\geometry{margin=1in}

% --- Custom Commands ---
\newcommand{\F}{\mathbb{F}}
\newcommand{\R}{\mathbb{R}}
\newcommand{\C}{\mathbb{C}}
\newcommand{\Q}{\mathbb{Q}}
\newcommand{\Z}{\mathbb{Z}}
\newcommand{\N}{\mathbb{N}}
\newcommand{\eps}{\varepsilon}
\newcommand{\del}{\partial}

\DeclareMathOperator{\Span}{span}
\DeclareMathOperator{\Ker}{Ker}
\DeclareMathOperator{\Img}{Im}
\DeclareMathOperator{\rank}{rank}
\DeclareMathOperator{\diag}{diag}

% --- Theorem Environments ---
\theoremstyle{definition}
\newtheorem{definition}{Definition}[section]

\theoremstyle{plain}
\newtheorem{theorem}[definition]{Theorem}
\newtheorem{lemma}[definition]{Lemma}
\newtheorem{proposition}[definition]{Proposition}
\newtheorem{corollary}[definition]{Corollary}

\theoremstyle{remark}
\newtheorem{remark}[definition]{Remark}
\newtheorem{example}[definition]{Example}

% --- Title Info ---
\title{Mathematics for Machine Learning}
\author{Leo Baker-Hytch}
\date{\today}

% --- Begin Document ---
\begin{document}

\maketitle
\tableofcontents
\vspace{1em}

\section{Vector spaces and subspaces}

\begin{definition}
    A \emph{vector space} over a field \( \F \) is a set \( V \) equipped with two operations: vector addition and
    scalar multiplication, satisfying the usual axioms.
\end{definition}

\begin{definition}
    A subset \( U \subseteq V \) is a \emph{subspace} of a vector space \( V \) if:
    \begin{itemize}
        \item \( 0 \in U \)
        \item \( u, w \in U \Rightarrow u + w \in U \)
        \item \( a \in \mathbb{F}, u \in U \Rightarrow a u \in U \)
    \end{itemize}
\end{definition}

\begin{example}
    The set of continuous real-valued functions \( C(\R) \) is a subspace of \( \R^\R \).
\end{example}

\section{Function spaces}

Let \( \R^\R \) denote the set of all functions from \( \R \) to \( \R \). We define:

\begin{itemize}
    \item \( C(\R) \): continuous functions
    \item \( C^1(\R) \): continuously differentiable functions
    \item \( C^\infty(\R) \): smooth functions
\end{itemize}

Clearly, \( C^\infty(\R) \subset C^1(\R) \subset C(\R) \subset \R^\R \).

\section{Sequence spaces}

Let \( \C^\N \) denote the set of all complex sequences.

\begin{itemize}
    \item \( c_0 \): sequences converging to zero
    \item \( \ell^\infty \): bounded sequences
    \item \( \ell^2 \): square-summable sequences
\end{itemize}

Then we have: \( c_0 \subset \ell^\infty \subset \C^\N \), and each is a subspace of the next.

\end{document}
