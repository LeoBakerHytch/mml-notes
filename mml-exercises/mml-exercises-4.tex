\documentclass[11pt]{article}

% Input encoding
\usepackage[utf8]{inputenc}

% Geometry
\usepackage{geometry}
\geometry{margin=1in}
\setlength{\parskip}{1em} % Vertical space between paragraphs
\setlength{\parindent}{0pt} % Remove indentation

% Subliminal refinements towards typographical perfection
\usepackage[tracking=true]{microtype}

% Subtitle
\usepackage{titling}

% Good old Computer Modern
\usepackage{amsfonts}
\usepackage{amssymb}
\usepackage{bm}
\usepackage[T1]{fontenc}

% Mathematical typesetting
\usepackage{amsmath} % Base
\usepackage{amsthm} % Theorems
\usepackage{mathtools} % Tools to use with amsmath
\usepackage{xfrac} % Split-level fractions

% Generic symbols
\usepackage{textcomp}
\usepackage{gensymb}

% Plotting
\usepackage{tikz}            % Plots
\usepackage{tikz-3dplot}     % 3D plots
\usetikzlibrary{arrows.meta, positioning}

% Hyperlinks
\usepackage[
    colorlinks=true,
    linkcolor=blue,
    urlcolor=blue,
    citecolor=blue
]{hyperref}

% Custom packages
\usepackage{../common/my-headings}
\usepackage{../common/my-list-control}
\usepackage{../common/my-math-macros}
\usepackage{../common/my-matrix-control}
\usepackage{../common/my-subtitle}
\usepackage{../common/my-theorem-envs}


\title{Mathematics for Machine Learning}
\subtitle{Exercises 4: Matrix decompositions}
\author{}
\date{}

\begin{document}

\maketitle

\begin{enumerate}

    \item[4.1] Compute the determinant using the Laplace expansion (using the first row) and the
          Sarrus rule for
          \[
              \mat{A} \coloneq
              \begin{bmatrix}
                  1 & 3 & 5 \\
                  2 & 4 & 6 \\
                  0 & 2 & 4 \\
              \end{bmatrix} \! .
          \]
          \vspace{1em}
          Using the Laplace expansion:
          \[
              \det(\mat{A})
              = 1
              \begin{vmatrix}
                  4 & 6 \\
                  2 & 4 \\
              \end{vmatrix}
              - 3
              \begin{vmatrix}
                  2 & 6 \\
                  0 & 4 \\
              \end{vmatrix}
              + 5
              \begin{vmatrix}
                  2 & 4 \\
                  0 & 2 \\
              \end{vmatrix}
              = (1 \cdot 4) + (-3 \cdot 8) + (5 \cdot 4)
              = 0.
          \]
          Using the Sarrus rule:
          \[
              \begin{aligned}
                  \det(\mat{A}) & = (1 \cdot 4 \cdot 4) + (2 \cdot 2 \cdot 5) + (0 \cdot 3 \cdot 6) \\
                                & - (0 \cdot 4 \cdot 5) - (1 \cdot 2 \cdot 6) - (2 \cdot 3 \cdot 4) \\
                                & = 16 + 20 + 0 - 0 - 12 - 24                                       \\
                                & = 0.
              \end{aligned}
          \]

    \item[4.2] Compute the following determinant efficiently:
          \[
              \begin{aligned}
                  \begin{vmatrix}
                      \begin{array}{rrrrr}
                          2  & 0  & 1 & 2  & 0 \\
                          2  & -1 & 0 & 1  & 1 \\
                          0  & 1  & 2 & 1  & 2 \\
                          -2 & 0  & 2 & -1 & 2 \\
                          2  & 0  & 0 & 1  & 1 \\
                      \end{array}
                  \end{vmatrix}
                  \; = \;
                  \begin{vmatrix}
                      \begin{array}{rrrrr}
                          2 & 0  & 1  & 2  & 0  \\
                          0 & -1 & -1 & -1 & 1  \\
                          0 & 0  & 1  & 0  & 3  \\
                          0 & 0  & 0  & 1  & -7 \\
                          0 & 0  & 0  & 0  & -3 \\
                      \end{array}
                  \end{vmatrix}
                  \; = \;
                  6.
              \end{aligned}
          \]

    \item[4.3] Compute the eigenspaces of
          \[
              \text{a.} \quad
              \mat{A} \coloneq
              \begin{bmatrix}
                  1 & 0 \\
                  1 & 1 \\
              \end{bmatrix}
              \qquad
              \text{b.} \quad
              \mat{B} \coloneq
              \begin{bmatrix*}[r]
                  -2 & 2 \\
                  2 & 1 \\
              \end{bmatrix*}
          \]

          \begin{enumerate}
              \item [a.] Characteristic polynomial of $\mat{A}$:
                    \[
                        p_{\mat{A}}(\lambda)
                        = \det(\mat{A} - \lambda \mat{I})
                        = (\lambda - 1)^2
                        = 0
                        \quad \implies \quad
                        \lambda \in \set{1}.
                    \]
                    Gives:
                    \[
                        E_1 = \set{ \x \in \R^2 \mid \mat{A} \x = \x } = \Span \args{
                            \begin{bmatrix}
                                0 \\ 1
                            \end{bmatrix}
                        }.
                    \]

              \item [b.] Characteristic polynomial of $\mat{B}$:
                    \[
                        \begin{aligned}
                            p_{\mat{B}}(\lambda)
                             & = \det(\mat{B} - \lambda \mat{I}) \\
                             & = (-2 - \lambda)(\lambda - 1) - 4 \\
                             & = \lambda^2 + \lambda - 6         \\
                             & = (\lambda + 3)(\lambda - 2)      \\
                             & = 0
                        \end{aligned}
                        \quad \implies \quad
                        \lambda \in \set{2, -3}.
                    \]
                    Gives:
                    \[
                        \begin{aligned}
                            E_2
                            = \Span \args{
                                \begin{bmatrix}
                                    1 \\ 2
                                \end{bmatrix}
                            },
                        \end{aligned}
                        \quad
                        \begin{aligned}
                            E_{-3}
                            = \Span \args{
                                \begin{bmatrix*}[r]
                                    2 \\ -1
                                \end{bmatrix*}
                            }.
                        \end{aligned}
                    \]
          \end{enumerate}

    \item[4.4] Compute all eigenspaces of:
          \[
              \mat{A} \coloneq
              \begin{bmatrix*}[r]
                  0 & -1 & 1 & 1 \\
                  -1 & 1 & -2 & 3 \\
                  2 & -1 & 0 & 0 \\
                  1 & -1 & 1 & 0 \\
              \end{bmatrix*}
          \]

          Characteristic polynomial:
          \[
              \det(\mat{A} - \lambda \mat{I}) = (\lambda - 2)(\lambda - 1)(\lambda + 1)^2 = 0
              \quad \implies \quad
              \lambda \in \set{2, 1, -1}.
          \]
          \[
              E_2 = \Span \args{
                  \begin{bmatrix}
                      1 \\ 0 \\ 1 \\ 1
                  \end{bmatrix}
              }
              , \quad
              E_1 = \Span \args{
                  \begin{bmatrix}
                      1 \\ 1 \\ 1 \\ 1
                  \end{bmatrix}
              }
              , \quad
              E_{-1} = \Span \args{
                  \begin{bmatrix}
                      0 \\ 1 \\ 1 \\ 0
                  \end{bmatrix}
              }
              .
          \]

    \item[4.5] Diagonalizability of a matrix is unrelated to its invertibility.  Determine for the
          following four matrices whether they are diagonalizable and/or invertible:
          \[
              \mat{I_2} \coloneq
              \begin{bmatrix}
                  1 & 0 \\
                  0 & 1 \\
              \end{bmatrix}
              \quad
              \mat{B} \coloneq
              \begin{bmatrix}
                  1 & 0 \\
                  0 & 0 \\
              \end{bmatrix}
              \quad
              \mat{C} \coloneq
              \begin{bmatrix}
                  1 & 1 \\
                  0 & 1 \\
              \end{bmatrix}
              \quad
              \mat{D} \coloneq
              \begin{bmatrix}
                  0 & 1 \\
                  0 & 0 \\
              \end{bmatrix}
          \]
          \begin{enumerate}[align=left]
              \item[$\mat{I}_2$] Invertible and diagonalizable (already diagonal).
              \item[$\mat{B}$] Diagonalizable but not invertible (already diagonal).
              \item[$\mat{C}$] Invertible but not diagonalizable (not a full basis of eigenvectors).
              \item[$\mat{D}$] Neither invertible nor diagonalizable (not a full basis of eigenvectors).
          \end{enumerate}

          \pagebreak

    \item[4.6] Compute the eigenspaces of the following transformation matrices.  Are they
          diagonalizable?
          \[
              \mat{A} \coloneq
              \begin{bmatrix}
                  2 & 3 & 0 \\
                  1 & 4 & 3 \\
                  0 & 0 & 1 \\
              \end{bmatrix}
              \qquad
              \mat{B} \coloneq
              \begin{bmatrix}
                  1 & 1 & 0 & 0 \\
                  0 & 0 & 0 & 0 \\
                  0 & 0 & 0 & 0 \\
                  0 & 0 & 0 & 0 \\
              \end{bmatrix}
          \]
          \begin{enumerate}
              \item[a.] Characteristic polynomial of $\mat{A}:$
                    \[
                        \begin{aligned}
                            p_{\mat{A}}(\lambda)
                             & = \det(\mat{A} - \lambda \mat{I}_3)             \\
                             & = (\lambda - 1)((\lambda - 2)(\lambda - 4) - 3) \\
                             & = (\lambda - 1)(\lambda^2 - 6 \lambda + 5)      \\
                             & = (\lambda - 1)^2(\lambda - 5)                  \\
                             & = 0
                        \end{aligned}
                        \quad \implies \quad
                        \lambda \in \set{5, 1}.
                    \]
                    \[
                        \begin{aligned}
                            E_5 = \set{ \x \in \R^3 \mid \mat{A} \x = 5 \x } = \Span \args{
                                \begin{bmatrix}
                                    1 \\ 1 \\ 0
                                \end{bmatrix}
                            }
                            \! ,
                            \\[1em]
                            E_{1} = \set{ \x \in \R^3 \mid \mat{A} \x = \x } = \Span \args{
                                \begin{bmatrix*}[r]
                                    3 \\ -1 \\ 0
                                \end{bmatrix*}
                            }
                            \! .
                        \end{aligned}
                    \]
                    $\mat{A}$ is not diagonalizable, since it lacks a full basis of eigenvectors
                    (specifically, $\rank(\mat{A} - \mat{I}) = 2$, so the geometric multiplicity of
                    $\lambda_2 = 0$ is less than its algebraic multiplicity).

              \item[b.] Characteristic polynomial of $\mat{B}$:
                    \[
                        \begin{aligned}
                            p_{\mat{B}}
                             & = \det(\mat{B} - \lambda \mat{I}_4)           \\
                             & = (-\lambda)(-\lambda)(-\lambda)(\lambda - 1) \\
                             & = \lambda^3 (\lambda - 1)                     \\
                             & = 0
                        \end{aligned}
                        \quad \implies \quad
                        \lambda \in \set{1, 0}.
                    \]
                    \[
                        E_1 = \Span \args{
                            \begin{bmatrix}
                                1 \\ 0 \\ 0 \\ 0
                            \end{bmatrix}
                        }
                        \! ,
                        \qquad
                        E_0 = \Span \args{
                            \begin{bmatrix*}[r]
                                1 \\ -1 \\ 0 \\ 0
                            \end{bmatrix*}
                            \! ,
                            \begin{bmatrix}
                                0 \\ 0 \\ 1 \\ 0
                            \end{bmatrix}
                            \! ,
                            \begin{bmatrix}
                                0 \\ 0 \\ 0 \\ 1
                            \end{bmatrix}
                        }
                        \! .
                    \]
                    $\mat{B}$ is diagonalizable, since it has a full basis of eigenvectors:
                    \[
                        \mat{B} = \mat{P} \mat{D} \mat{P}^{-1} =
                        \begin{bmatrix*}[r]
                            1 & 1  & 0 & 0 \\
                            0 & -1 & 0 & 0 \\
                            0 & 0  & 1 & 0 \\
                            0 & 0  & 0 & 1 \\
                        \end{bmatrix*}
                        \begin{bmatrix}
                            1 & 0 & 0 & 0 \\
                            0 & 0 & 0 & 0 \\
                            0 & 0 & 0 & 0 \\
                            0 & 0 & 0 & 0 \\
                        \end{bmatrix}
                        \begin{bmatrix*}[r]
                            1 & 1  & 0 & 0 \\
                            0 & -1 & 0 & 0 \\
                            0 & 0  & 1 & 0 \\
                            0 & 0  & 0 & 1 \\
                        \end{bmatrix*}
                        \! .
                    \]
          \end{enumerate}

          \pagebreak

    \item[4.7] Are the following matrices diagonalizable?  If yes, determine their diagonal
          form, and a basis with respect to which the transformation matrices are diagonal.  If no,
          give reasons why they are not diagonalizable.
          \[
              \mat{A} \coloneq
              \begin{bmatrix*}[r]
                  0 & 1 \\
                  -8 & 4 \\
              \end{bmatrix*}
              \qquad
              \mat{B} \coloneq
              \begin{bmatrix}
                  1 & 1 & 1 \\
                  1 & 1 & 1 \\
                  1 & 1 & 1 \\
              \end{bmatrix}
              \qquad
              \mat{C} \coloneq
              \begin{bmatrix*}[r]
                  5 & 4 & 2 & 1 \\
                  0 & 1 & -1 & -1 \\
                  -1 & -1 & 3 & 0 \\
                  1 & 1 & -1 & 2 \\
              \end{bmatrix*}
              \qquad
              \mat{D} \coloneq
              \begin{bmatrix*}[r]
                  5  & -6 & -6 \\
                  -1 & 4  & 2  \\
                  3  & -6 & -4 \\
              \end{bmatrix*}
          \]

          \begin{enumerate}[align=left]
              \item[$\mat{A}$] Characteristic polynomial, $p_{\mat{A}}(\lambda) = \lambda^2 - 4
                        \lambda + 8$, has no real roots, so $\mat{A}$ has no (non-trivial) eigenvectors, and is
                    therefore not diagonalizable.

              \item[$\mat{B}$] Characteristic polynomial has two real roots, $p_{\mat{B}}(\lambda) =
                        \lambda^2(\lambda - 3) = 0 \implies \lambda \in \set{3, 0}$, with matching algebraic and
                    geometric multiplicity, so $\mat{B}$ diagonalizes.
                    \[
                        E_3 = \Span \args{
                            \begin{bmatrix*}[r]
                                1 \\ 1 \\ 1
                            \end{bmatrix*}
                        }
                        \! ,
                        \qquad
                        E_0 = \Span \args{
                            \begin{bmatrix*}[r]
                                1 \\ -1 \\ 0
                            \end{bmatrix*}
                            \! ,
                            \begin{bmatrix*}[r]
                                0 \\ 1 \\ -1
                            \end{bmatrix*}
                        }
                        =\Span \args{
                            \begin{bmatrix*}[r]
                                1 \\ -1 \\ 0
                            \end{bmatrix*}
                            \! ,
                            \begin{bmatrix*}[r]
                                1 \\ 1 \\ -2
                            \end{bmatrix*}
                        }
                        \! .
                    \]
                    These eigenvectors are mutually orthogonal, so we can normalize them:
                    \[
                        \mat{Q} \coloneq
                        \begin{bmatrix*}[r]
                            \sfrac{\sqrt{3}}{3} & \sfrac{\sqrt{2}}{2}  & \sfrac{\sqrt{6}}{6}  \\
                            \sfrac{\sqrt{3}}{3} & \sfrac{-\sqrt{2}}{2} & \sfrac{\sqrt{6}}{6}  \\
                            \sfrac{\sqrt{3}}{3} & 0                    & \sfrac{-\sqrt{6}}{3} \\
                        \end{bmatrix*}
                        \!,
                        \quad
                        \mat{\Lambda} \coloneq
                        \begin{bmatrix}
                            3 & 0 & 0 \\
                            0 & 0 & 0 \\
                            0 & 0 & 0 \\
                        \end{bmatrix}
                        \!,
                        \quad
                        \mat{B} = \mat{Q} \mat{\Lambda} \mat{Q}^\top.
                    \]

              \item[$\mat{C}$] Characteristic polynomial has three real roots, $p_\mat{C} = (\lambda
                        - 4)^2(\lambda - 2)(\lambda - 1) \implies \lambda \in \set{4, 2, 1}$.  The
                    corresponding eigenspaces are:
                    \[
                        E_4 = \Span \args{
                            \begin{bmatrix*}[r]
                                1 \\ 0 \\ -1 \\ 1
                            \end{bmatrix*}
                        }
                        \! , \quad
                        E_2 = \Span \args{
                            \begin{bmatrix*}[r]
                                1 \\ -1 \\ 0 \\ 1
                            \end{bmatrix*}
                        }
                        \! , \quad
                        E_1 = \Span \args{
                            \begin{bmatrix*}[r]
                                -1 \\ 1 \\ 0 \\ 0
                            \end{bmatrix*}
                        }.
                    \]
                    Since $\lambda_1 = 4$ has algebraic multiplicity 2, but geometric multiplicity
                    of only 1, $\mat{C}$ does not diagonalize (it lacks a full basis of
                    eigenvectors).

                    \pagebreak

              \item[$\mat{D}$] Characteristic polynomial has two real roots, $p_{\mat{D}} = (\lambda - 2)^2 (\lambda -
                        1) = 0 \implies \lambda \in \set{2, 1}$.  The corresponding eigenspaces are:
                    \[
                        E_2 = \Span \args{
                            \begin{bmatrix}
                                2 \\ 1 \\ 0
                            \end{bmatrix}
                            \! ,
                            \begin{bmatrix}
                                2 \\ 0 \\ 1
                            \end{bmatrix}
                        }
                        \! , \quad
                        E_1 = \Span \args{
                            \begin{bmatrix*}[r]
                                3 \\ -1 \\ 3
                            \end{bmatrix*}
                        }
                        \! .
                    \]
                    Using Gram-Schmidt, we can orthonormalize the eigenvectors:
                    \[
                        \begin{alignedat}{3}
                             &
                            \vect{e}_1
                            \coloneq
                            \begin{bmatrix}
                                2 \\ 1 \\ 0
                            \end{bmatrix}
                            \! ,
                            \qquad
                             &   &
                            \vect{u}_1
                            \coloneq
                            \vect{e}_1
                             &   &
                            \vect{q}_1
                            \coloneq
                            \frac{\vect{u_1}}{\norm{\vect{u_1}}}
                            ,
                            \\[1em]
                             &
                            \vect{e}_2
                            \coloneq
                            \begin{bmatrix}
                                2 \\ 0 \\ 1
                            \end{bmatrix}
                            \! ,
                            \qquad
                             &   &
                            \vect{u}_2
                            \coloneq
                            \vect{e}_2
                            -
                            \frac{ \vect{u}_1 \vect{u}_1^\top }{ \norm{\vect{u}_1}^2 }
                            \vect{e}_2
                            ,
                            \qquad
                             &   &
                            \vect{q}_2
                            \coloneq
                            \frac{\vect{u}_2}{\norm{\vect{u}_2}}
                            ,
                            \\[1em]
                             &
                            \vect{e}_3
                            \coloneq
                            \begin{bmatrix*}[r]
                                3 \\ -1 \\ 3
                            \end{bmatrix*}
                            \! ,
                            \qquad
                             &   &
                            \vect{u}_3
                            \coloneq
                            \vect{e}_3
                            -
                            \frac{ \vect{u}_1 \vect{u}_1^\top }{ \norm{\vect{u}_1}^2 }
                            \vect{e}_3
                            -
                            \frac{ \vect{u}_1 \vect{u}_1^\top }{ \norm{\vect{u}_1}^2 }
                            \vect{e}_3
                            ,
                            \qquad
                             &   &
                            \vect{q}_3
                            \coloneq
                            \frac{\vect{u}_3}{\norm{\vect{u}_3}}
                            .
                            \\[1em]
                        \end{alignedat}
                    \]
                    Concatenating the resulting vectors, $\vect{q}_i$, into the matrix $\mat{Q}$, gives us:
                    \[
                        \mat{Q}
                        \coloneq
                        \begin{bmatrix*}[r]
                            \sfrac{2 \sqrt{5}}{5} & \sfrac{2 \sqrt{15}}{15} & \sfrac{-1}{3} \\
                            \sfrac{\sqrt{5}}{5}   & \sfrac{-4\sqrt{5}}{15}  & \sfrac{2}{3} \\
                            0                     & \sfrac{\sqrt{5}}{3}     & \sfrac{2}{3} \\
                        \end{bmatrix*}
                        \! ,
                        \quad
                        \mat{\Lambda}
                        \coloneq
                        \begin{bmatrix}
                            2 & 0 & 0 \\
                            0 & 2 & 0 \\
                            0 & 0 & 1 \\
                        \end{bmatrix}
                        ,
                        \quad
                        \mat{D} \neq \mat{Q} \mat{\Lambda} \mat{Q}^\top.
                    \]
                    \dots aaand this does not work at all!

                    $\mat{D}$ is not a symmetric matrix.  Its eigenvectors are \emph{not} orthogonal
                    (unlike those of $\mat{B}$ earlier, which only required normalizing to form an
                    ONB).  Gram-Schmidt changes the directions of the basis, so the eigenvectors are
                    no longer eigenvectors.

                    Defining instead the eigenbasis matrix, $\mat{P}$, we have:
                    \[
                        \mat{P}
                        \coloneq
                        \begin{bmatrix*}[r]
                            2 & 2 & 3 \\
                            1 & 0 & -1 \\
                            0 & 1 & 3 \\
                        \end{bmatrix*}
                        \! ,
                        \qquad
                        \mat{D} = \mat{P} \mat{\Lambda} \mat{P}^{-1}
                        =
                        \begin{bmatrix*}[r]
                            5  & -6 & -6 \\
                            -1 & 4  & 2  \\
                            3  & -6 & -4 \\
                        \end{bmatrix*}
                        \! .
                    \]

          \end{enumerate}

          \begin{center}
              \scriptsize
              \makebox[\textwidth][c]{%
                  \renewcommand{\arraystretch}{1.5}
                  \begin{tabular}{|c|c|c|c|c|}
                      \hline
                      Decomp.                                                  & $\mat{A} =$ & Conditions & Factor properties & Use case \\
                      \hline
                      Spectral                                                 &
                      $\mat{Q} \mat{\Lambda} \mat{Q}^\top$                     &
                      $\mat{A}$ symmetric                                      &
                      $\mat{Q}^\top \mat{Q} = \mat{I}, \; \mat{\Lambda}$ diag. &
                      PCA, covariance, normal matrices                                                                                   \\
                      \hline
                      Eigen                                                    &
                      $\mat{P} \! \mat{\Lambda} \mat{P}^{-1}$                  &
                      $\mat{A}$ diagonalizable                                 &
                      $\mat{P}$ invertible (\emph{not} orth.)                  &
                      General diagonalization                                                                                            \\
                      \hline
                      Schur                                                    &
                      $\mat{Q} \mat{T} \mat{Q}^\top$                           &
                      Any square $\mat{A}$                                     &
                      $\mat{Q}$ orth., $\mat{T}$ upper tri.                    &
                      Numerical stability, QR algorithm                                                                                  \\
                      \hline
                      SVD                                                      &
                      $\mat{U} \mat{\Sigma} \mat{V}^\top$                      &
                      Any $m \times n$                                         &
                      $\mat{U}, \mat{V}$ orth., $\mat{\Sigma}$ diag. ($\ge 0$) &
                      Data comp., low-rank approx.                                                                                       \\
                      \hline
                      QR                                                       &
                      $\mat{Q} \mat{R}$                                        &
                      Any full-rank $m \times n$                               &
                      $\mat{Q}^\top \mat{Q} = \mat{I}, \; \mat{R}$ upper tri.  &
                      Least squares, solving $\mat{A}\vect{x}=\vect{b}$                                                                  \\
                      \hline
                      LU                                                       &
                      $\mat{L} \mat{U}$                                        &
                      Any square, non-sing. $\mat{A}$                          &
                      $\mat{L}$ l-tri. (unit diag.), $\mat{U}$ u-tri.          &
                      Fast solves for multiple $\vect{b}$, basis updates                                                                 \\
                      \hline
                      Cholesky                                                 &
                      $\mat{L} \mat{L}^\top$                                   &
                      $\mat{A}$ symmetric positive definite                    &
                      $\mat{L}$ lower tri.                                     &
                      Cov. matrices, Gaussian procs, optim.                                                                              \\
                      \hline
                  \end{tabular}%
              }
          \end{center}

          \pagebreak

          Let's instead do a QR decomposition, $\mat{D} = \mat{Q} \mat{R}$, where $\mat{Q}$ is
          orthogonal and $\mat{R}$ is upper-triangular.  Normalizing the $\vect{q}_i$ as we go, the
          squared norm denominator disappears (since by construction $\norm{\vect{q}_i}^2 = 1$).
          Also, in the projection, we avoid an intermediate $3 \times 3$ matrix ($\vect{q}_i
              \vect{q}_i^\top$), by computing the residual coefficients $r_{ij}$ (which we store at the
          specified indices in $\mat{R}$, which is initialized with all zeros).  The normalized
          column vectors $\vect{q}_i$ are concatenated, $\mat{Q} = [ \vect{q}_1 \; \vect{q}_2 \;
              \vect{q}_3 ]$.

          We compute the following in floating-point arithmetic, rather than exactly, since the
          normalization results in nested roots that quickly become extremely unwieldy:
          \[
              \begin{alignedat}{3}
                   &
                  \vect{d}_1
                  \coloneq
                  \begin{bmatrix*}[r]
                      5 \\ -1 \\ 3
                  \end{bmatrix*}
                  \! ,
                  \qquad
                   &   &
                  \begin{aligned}{}
                       &
                      \vect{u}_1
                      \coloneq
                      \vect{d}_1,
                      \\
                       &
                      r_{11} \coloneq \norm{\vect{u_1}},
                  \end{aligned}
                   &   &
                  \vect{q}_1
                  \coloneq
                  \frac{\vect{u_1}}{r_{11}}
                  ,
                  \\[1em]
                   &
                  \vect{d}_2
                  \coloneq
                  \begin{bmatrix*}[r]
                      -6 \\ 4 \\ -6
                  \end{bmatrix*}
                  \! ,
                  \qquad
                   &   &
                  \begin{aligned}
                       &
                      r_{12} \coloneq \vect{q}_1^\top \vect{d}_2,
                      \\
                       &
                      \vect{u}_2
                      \coloneq
                      \vect{d}_2
                      - r_{12} \vect{q}_1,
                      \\
                       &
                      r_{22} \coloneq \norm{\vect{u}_2},
                  \end{aligned}
                  \qquad
                   &   &
                  \vect{q}_2
                  \coloneq
                  \frac{\vect{u}_2}{r_{22}}
                  ,
                  \\[1em]
                   &
                  \vect{d}_3
                  \coloneq
                  \begin{bmatrix*}[r]
                      -6 \\ 2 \\ -4
                  \end{bmatrix*}
                  \! ,
                  \qquad
                   &   &
                  \begin{aligned}
                       &
                      r_{13} \coloneq \vect{q}_1^\top \vect{d}_3,
                      \quad
                      r_{23} \coloneq \vect{q}_2^\top \vect{d}_3,
                      \\
                       &
                      \vect{u}_3
                      \coloneq
                      \vect{d}_3
                      - r_{13} \vect{q}_1
                      - r_{23} \vect{q}_2,
                      \\
                       &
                      r_{33} \coloneq \norm{\vect{u}_3},
                  \end{aligned}
                  \qquad
                   &   &
                  \vect{q}_3
                  \coloneq
                  \frac{\vect{u}_3}{r_{33}}
                  .
                  \\[1em]
              \end{alignedat}
          \]
          We find,
          \[
              \mat{Q} \approx
              \begin{bmatrix*}[r]
                  -0.8452 & -0.4359 & -0.3094 \\
                  0.1690 & -0.7671 & 0.6189 \\
                  -0.5071 & 0.4707 & 0.7220 \\
              \end{bmatrix*}
              \! ,
              \qquad
              \mat{R} \approx
              \begin{bmatrix*}[r]
                  -5.9161 & 8.7896 & 7.4374 \\
                  0 & -3.2776 & -0.8020 \\
                  0 & 0 & 0.2063 \\
              \end{bmatrix*}
              \! .
          \]
          Computing $\mat{R} \mat{Q}$ is the first step of the QR algorithm, and we see the
          eigenvalues start to emerge on the diagonal (compare to those we calculated exactly
          before):
          \[
              \mat{R} \mat{Q} \approx
              \begin{bmatrix*}[r]
                  2.71  & -0.66 & 12.64 \\
                  -0.15 & 2.14  & -2.61 \\
                  -0.10 & 0.10  & 0.15
              \end{bmatrix*}
              \! .
          \]
          Since $\mat{R} \mat{Q} = \mat{Q}^\top \mat{D} \mat{Q}$, and we know $\mat{Q}$ is
          orthogonal by construction, this result is \emph{similar} to $\mat{D}$: it has the same
          eigenvalues.

          \pagebreak

    \item[4.8] Find the singular value decomposition (SVD) of the matrix
          \[
              \mat{A} \coloneq
              \begin{bmatrix*}[r]
                  3 & 2 & 2 \\
                  2 & 3 & -2 \\
              \end{bmatrix*}
          \]

          Compute the Gram matrix in the domain space:
          \[
              \mat{A}^\top \mat{A} =
              \begin{bmatrix*}[r]
                  13 & 12 & 2 \\
                  12 & 13 & -2 \\
                  2 & -2 & 8 \\
              \end{bmatrix*}
          \]
          Solve the characteristic polynomial:
          \[
              \begin{aligned}
                             & \det(\mat{A}^\top \mat{A} - \lambda \mat{I}) = 0 \\
                  \iff \quad & \lambda (\lambda - 9)(\lambda - 25) = 0          \\
                  \iff \quad & \lambda \in \set{25, 9, 0}.
              \end{aligned}
          \]
          Gram matrix eigenvalues are the squared singular values, $\sigma_i$, so
          \[
              \sigma_1 = 5,
              \quad
              \sigma_2 = 3,
              \quad
              \sigma_3 = 0.
          \]
          Normalizing the Gram matrix eigenvectors gives the right singular vectors, $\vect{v}_i$,
          \[
              \vect{v}_1 =
              \begin{bmatrix}
                  \sfrac{\sqrt{2}}{2} \\ \sfrac{\sqrt{2}}{2} \\ 0
              \end{bmatrix} \! ,
              \quad
              \vect{v}_2 =
              \begin{bmatrix*}[r]
                  \sfrac{\sqrt{2}}{6} \\ \sfrac{-\sqrt{2}}{6} \\ \sfrac{2\sqrt{2}}{3}
              \end{bmatrix*} \! ,
              \quad
              \vect{v}_3 =
              \begin{bmatrix*}[r]
                  \sfrac{2}{3} \\ \sfrac{-2}{3} \\ \sfrac{-1}{3}
              \end{bmatrix*} \! .
          \]
          Given $\mat{A} = \mat{U} \mat{\Sigma} \mat{V}^\top$, find the left singular vectors,
          \[
              \mat{U} = \mat{A} \mat{V} \mat{\Sigma}^+ =
              \begin{bmatrix*}[r]
                  3 & 2 & 2 \\
                  2 & 3 & -2 \\
              \end{bmatrix*}
              \begin{bmatrix*}[r]
                  \sfrac{\sqrt{2}}{2} & \sfrac{\sqrt{2}}{6}  & \sfrac{2}{3}  \\
                  \sfrac{\sqrt{2}}{2} & \sfrac{-\sqrt{2}}{6} & \sfrac{-2}{3} \\
                  0                   & \sfrac{2\sqrt{2}}{3} & \sfrac{-1}{3} \\
              \end{bmatrix*}
              \begin{bmatrix*}[c]
                  \sfrac{1}{5} & 0 \\
                  0 & \sfrac{1}{3} \\
                  0 & 0 \\
              \end{bmatrix*}
              =
              \begin{bmatrix*}[r]
                  \sfrac{\sqrt{2}}{2} & \sfrac{\sqrt{2}}{2} \\
                  \sfrac{\sqrt{2}}{2} & \sfrac{-\sqrt{2}}{2} \\
              \end{bmatrix*}
          \]
          So the full decomposition,
          \[
              \mat{A} = \mat{U} \mat{\Sigma} \mat{V}^\top =
              \begin{bmatrix*}[r]
                  \sfrac{\sqrt{2}}{2} & \sfrac{\sqrt{2}}{2} \\
                  \sfrac{\sqrt{2}}{2} & \sfrac{-\sqrt{2}}{2} \\
              \end{bmatrix*}
              \begin{bmatrix}
                  5 & 0 & 0 \\
                  0 & 3 & 0 \\
              \end{bmatrix}
              \begin{bmatrix*}[r]
                  \sfrac{\sqrt{2}}{2} & \sfrac{\sqrt{2}}{2}  & 0 \\
                  \sfrac{\sqrt{2}}{6} & \sfrac{-\sqrt{2}}{6} & \sfrac{2\sqrt{2}}{3} \\
                  \sfrac{2}{3}        & \sfrac{-2}{3}        & \sfrac{-1}{3} \\
              \end{bmatrix*}
              \! .
          \]

          \pagebreak

    \item[4.9] Find the SVD of
          \[
              \mat{A} \coloneq
              \begin{bmatrix*}[r]
                  2 & 2 \\
                  -1 & 1 \\
              \end{bmatrix*}
          \]

    \item[4.10] Find the rank-1 approximation of
          \[
              \mat{A} \coloneq
              \begin{bmatrix*}[r]
                  3 & 2 & 2 \\
                  2 & 3 & -2 \\
              \end{bmatrix*}
          \]

    \item[4.11] Show that for any $\mat{A} \in \R^{m \times n}$ the matrices $\mat{A}^\top
              \mat{A}$ and $\mat{A} \mat{A}^\top$ possess the same non-zero eigenvalues.

    \item[4.12] Show that for $\vect{x} \neq 0$, Theorem 4.24 holds.  That is,
          \[
              \max_{\vect{x}} \frac{ \norm{\mat{A} \vect{x}}_2 }{ \norm{\vect{x}}_2 } = \sigma_1,
          \]
          where $\sigma_1$ is the largest singular value of $\mat{A} \in \R^{m \times n}$.

\end{enumerate}

\end{document}
