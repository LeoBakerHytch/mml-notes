\documentclass[11pt]{article}

% --- Packages ---
\usepackage{mathpazo}        % Loads Palatino + math support
\usepackage[utf8]{inputenc}
\usepackage{amsmath, amssymb, amsthm, amsfonts}
\usepackage{mathtools}       % For extra math tools
\usepackage{bm}              % Bold math symbols
\usepackage{enumitem}        % Better control over lists
\usepackage{geometry}        % Better margins
\usepackage[colorlinks=true, linkcolor=blue, urlcolor=blue, citecolor=blue]{hyperref}
\geometry{margin=1in}

% --- Custom commands ---
\newcommand{\F}{\mathbb{F}}
\newcommand{\R}{\mathbb{R}}
\newcommand{\C}{\mathbb{C}}
\newcommand{\Q}{\mathbb{Q}}
\newcommand{\Z}{\mathbb{Z}}
\newcommand{\N}{\mathbb{N}}

\newcommand{\G}{\mathcal{G}}

\newcommand{\Xcal}{\mathcal{X}}  % Input space
\newcommand{\Ycal}{\mathcal{Y}}  % Output / label space
\newcommand{\Hcal}{\mathcal{H}}  % Hypothesis space
\newcommand{\Dcal}{\mathcal{D}}  % Distribution or dataset

\newcommand{\eps}{\varepsilon}
\newcommand{\del}{\partial}

\newcommand{\vect}[1]{\bm{#1}}  % Vector
\newcommand{\mat}[1]{\bm{#1}}   % Matrix

\newcommand{\abs}[1]{\left|#1\right|}                    % Absolute value
\newcommand{\norm}[1]{\left\lVert#1\right\rVert}         % Norm
\newcommand{\set}[1]{\left\{#1\right\}}                  % Generic set
\newcommand{\inner}[2]{\left\langle#1, #2\right\rangle}  % Inner product
\newcommand{\cls}[1]{\overline{#1}}                      % Congruence class

\DeclareMathOperator{\Span}{span}
\DeclareMathOperator{\Ker}{Ker}
\DeclareMathOperator{\Img}{Im}
\DeclareMathOperator{\rank}{rank}
\DeclareMathOperator{\diag}{diag}

% --- Theorem environments ---
\theoremstyle{definition}
\newtheorem{definition}{Definition}[section]

\theoremstyle{plain}
\newtheorem{theorem}[definition]{Theorem}
\newtheorem{lemma}[definition]{Lemma}
\newtheorem{proposition}[definition]{Proposition}
\newtheorem{corollary}[definition]{Corollary}

\theoremstyle{remark}
\newtheorem{remark}[definition]{Remark}
\newtheorem{example}[definition]{Example}

% --- Title Info ---
\title{Mathematics for Machine Learning}
\author{}
\date{}

% --- Begin Document ---
\begin{document}

\maketitle
\vspace{1em}

\setcounter{section}{1}
\section{Linear algebra}

\begin{enumerate}

    \item[\textbf{2.1}]

          We consider \(\left(\R \setminus \{-1\}, \star \right)\), where:
          \[
              a \star b = ab + a + b \qquad a, b \in \R \setminus \{-1\}.
          \]

    \item[a.] Show that \(\left(\R \setminus \{-1\}, \star \right)\) is an Abelian group.

          \subsubsection*{Neutral element}

          We have \( 0 \in \R \setminus \{-1\} \), and for all \(a \in \R \setminus \{-1\}\):
          \[
              \begin{aligned}
                  a \star 0 = a0 + a + 0 = a, & \quad \textrm{and} \\
                  0 \star a = 0a + 0 + a = a.
              \end{aligned}
          \]

          \subsubsection*{Commutativity}

          For all \(a, b \in \R \setminus \{-1\}\), we have:
          \[
              \begin{aligned}
                  a \star b & = ab + a + b \\
                            & = ba + b + a \\
                            & = b \star a.
              \end{aligned}
          \]

          \subsubsection*{Associativity}

          For all \(a, b, c \in \R \setminus \{-1\}\), we have:
          \[
              \begin{aligned}
                  (a \star b) \star c & = (ab + a + b) \star c               \\
                                      & = (abc + ac + bc) + (ab + a + b) + c \\
                                      & = a (bc + b + c) + a + (bc + b + c)  \\
                                      & = a (b \star c) + a + (b \star c)    \\
                                      & = a \star (b \star c).
              \end{aligned}
          \]

          \subsubsection*{Existence of inverse}

          For all \(a \in \R \setminus \{-1\}\), we require the existence of an element \(b\) such that:
          \[
              \begin{alignedat}{3}
                            &  & a \star b = b \star a & \; =\; &  & 0                 \\
                  \iff\quad &  & ab + a + b            & \; =\; &  & 0                 \\
                  \iff\quad &  & b(a + 1) + a          & \; =\; &  & 0                 \\
                  \iff\quad &  & b                     & \; =\; &  & \dfrac{-a}{a + 1}
              \end{alignedat}
          \]
          This expression for \(b\) is always defined, since \(a\) cannot be \(-1\), and the denominator is always non-zero.

          \subsubsection*{Closure under \(\star\)}

          For contradiction, assume that there exist \(a, b \in \R \setminus \{-1\}\), such that:
          \[
              \begin{alignedat}{3}
                            &  & a \star b  & \; = \; &  & -1                    \\
                  \iff\quad &  & ab + a + b & \; = \; &  & -1                    \\
                  \iff\quad &  & a (1 + b)  & \; = \; &  & - (1 + b)             \\
                  \iff\quad &  & a          & \; = \; &  & -\dfrac{1 + b}{1 + b} \\
                  \iff\quad &  & a          & \; = \; &  & -1.                   \\
              \end{alignedat}
          \]

    \item[b.] In the Abelian group \(\left(\R \setminus \{-1\}, \star \right)\), solve
          \[
              3 \star x \star x = 15.
          \]

          \subsubsection*{Solution}

          We have
          \[
              \begin{alignedat}{3}
                            &  & 3 \star x \star x          &  & \; = \; & 15 \\
                  \iff\quad &  & (3x + 3 + x) \star x       &  & \; =\;  & 15 \\
                  \iff\quad &  & (4x + 3) \star x           &  & \; =\;  & 15 \\
                  \iff\quad &  & (4x^2 + 3x) + (4x + 3) + x &  & \; =\;  & 15 \\
                  \iff\quad &  & 4x^2 + 8x                  &  & \; =\;  & 12 \\
                  \iff\quad &  & x^2 + 2x -3                &  & \; =\;  & 0  \\
                  \iff\quad &  & (x + 3) (x - 1)            &  & \; =\;  & 0
              \end{alignedat}
          \]
          which yields the solutions \(x \in \{1, -3\} \subset \R \setminus \{-1\}\).

\end{enumerate}

\end{document}
