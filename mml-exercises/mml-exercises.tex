\documentclass[11pt]{article}

% --- Packages ---
\usepackage{mathpazo}        % Loads Palatino + math support
\usepackage[utf8]{inputenc}
\usepackage{amsmath, amssymb, amsthm, amsfonts}
\usepackage{mathtools}       % For extra math tools
\usepackage{bm}              % Bold math symbols
\usepackage{enumitem}        % Better control over lists
\usepackage{geometry}        % Better margins
\usepackage[colorlinks=true, linkcolor=blue, urlcolor=blue, citecolor=blue]{hyperref}
\geometry{margin=1in}

% --- Custom commands ---
\newcommand{\F}{\mathbb{F}}
\newcommand{\R}{\mathbb{R}}
\newcommand{\C}{\mathbb{C}}
\newcommand{\Q}{\mathbb{Q}}
\newcommand{\Z}{\mathbb{Z}}
\newcommand{\N}{\mathbb{N}}

\newcommand{\G}{\mathcal{G}}

\newcommand{\Xcal}{\mathcal{X}}  % Input space
\newcommand{\Ycal}{\mathcal{Y}}  % Output / label space
\newcommand{\Hcal}{\mathcal{H}}  % Hypothesis space
\newcommand{\Dcal}{\mathcal{D}}  % Distribution or dataset

\newcommand{\eps}{\varepsilon}
\newcommand{\del}{\partial}

\newcommand{\vect}[1]{\bm{#1}}  % Vector
\newcommand{\mat}[1]{\bm{#1}}   % Matrix

\newcommand{\abs}[1]{\left|#1\right|}                    % Absolute value
\newcommand{\norm}[1]{\left\lVert#1\right\rVert}         % Norm
\newcommand{\set}[1]{\left\{#1\right\}}                  % Generic set
\newcommand{\inner}[2]{\left\langle#1, #2\right\rangle}  % Inner product
\newcommand{\cls}[1]{\overline{#1}}                      % Congruence class

\DeclareMathOperator{\Span}{span}
\DeclareMathOperator{\Ker}{Ker}
\DeclareMathOperator{\Img}{Im}
\DeclareMathOperator{\rank}{rank}
\DeclareMathOperator{\diag}{diag}

% --- Theorem environments ---
\theoremstyle{definition}
\newtheorem{definition}{Definition}[section]

\theoremstyle{plain}
\newtheorem{theorem}[definition]{Theorem}
\newtheorem{lemma}[definition]{Lemma}
\newtheorem{proposition}[definition]{Proposition}
\newtheorem{corollary}[definition]{Corollary}

\theoremstyle{remark}
\newtheorem{remark}[definition]{Remark}
\newtheorem{example}[definition]{Example}

% --- Title Info ---
\title{Mathematics for Machine Learning}
\author{}
\date{}

% --- Begin Document ---
\begin{document}

\maketitle
\vspace{1em}

\setcounter{section}{1}
\section{Linear algebra}

\begin{enumerate}

    \item[\textbf{2.1}]

          We consider \(\left(\R \setminus \{-1\}, \star \right)\), where:
          \[
              a \star b = ab + a + b \qquad a, b \in \R \setminus \{-1\}.
          \]

    \item[a.] Show that \(\left(\R \setminus \{-1\}, \star \right)\) is an Abelian group.

          \subsubsection*{Neutral element}

          We have \( 0 \in \R \setminus \{-1\} \), and for all \(a \in \R \setminus \{-1\}\):
          \[
              \begin{aligned}
                  a \star 0 = a0 + a + 0 = a, & \quad \textrm{and} \\
                  0 \star a = 0a + 0 + a = a.
              \end{aligned}
          \]

          \subsubsection*{Commutativity}

          For all \(a, b \in \R \setminus \{-1\}\), we have:
          \[
              \begin{aligned}
                  a \star b & = ab + a + b \\
                            & = ba + b + a \\
                            & = b \star a.
              \end{aligned}
          \]

          \subsubsection*{Associativity}

          For all \(a, b, c \in \R \setminus \{-1\}\), we have:
          \[
              \begin{aligned}
                  (a \star b) \star c & = (ab + a + b) \star c               \\
                                      & = (abc + ac + bc) + (ab + a + b) + c \\
                                      & = a (bc + b + c) + a + (bc + b + c)  \\
                                      & = a (b \star c) + a + (b \star c)    \\
                                      & = a \star (b \star c).
              \end{aligned}
          \]

          \subsubsection*{Existence of inverse}

          For all \(a \in \R \setminus \{-1\}\), we require the existence of an element \(b\) such that:
          \[
              \begin{alignedat}{3}
                            &  & a \star b = b \star a & \; =\; &  & 0                 \\
                  \iff\quad &  & ab + a + b            & \; =\; &  & 0                 \\
                  \iff\quad &  & b(a + 1) + a          & \; =\; &  & 0                 \\
                  \iff\quad &  & b                     & \; =\; &  & \dfrac{-a}{a + 1}
              \end{alignedat}
          \]
          This expression for \(b\) is always defined, since \(a\) cannot be \(-1\), and the denominator is always non-zero.

          \subsubsection*{Closure under \(\star\)}

          For contradiction, assume that there exist \(a, b \in \R \setminus \{-1\}\), such that:
          \[
              \begin{alignedat}{3}
                            &  & a \star b  & \; = \; &  & -1                    \\
                  \iff\quad &  & ab + a + b & \; = \; &  & -1                    \\
                  \iff\quad &  & a (1 + b)  & \; = \; &  & - (1 + b)             \\
                  \iff\quad &  & a          & \; = \; &  & -\dfrac{1 + b}{1 + b} \\
                  \iff\quad &  & a          & \; = \; &  & -1.                   \\
              \end{alignedat}
          \]

    \item[b.] In the Abelian group \(\left(\R \setminus \{-1\}, \star \right)\), solve
          \[
              3 \star x \star x = 15.
          \]

          \subsubsection*{Solution}

          We have
          \[
              \begin{alignedat}{3}
                            &  & 3 \star x \star x          &  & \; = \; & 15 \\
                  \iff\quad &  & (3x + 3 + x) \star x       &  & \; =\;  & 15 \\
                  \iff\quad &  & (4x + 3) \star x           &  & \; =\;  & 15 \\
                  \iff\quad &  & (4x^2 + 3x) + (4x + 3) + x &  & \; =\;  & 15 \\
                  \iff\quad &  & 4x^2 + 8x                  &  & \; =\;  & 12 \\
                  \iff\quad &  & x^2 + 2x -3                &  & \; =\;  & 0  \\
                  \iff\quad &  & (x + 3) (x - 1)            &  & \; =\;  & 0
              \end{alignedat}
          \]
          which yields the solutions \(x \in \{1, -3\} \subset \R \setminus \{-1\}\).

    \item[\textbf{2.1}]

          Let \(n\) be in \(\N \setminus \{0\}\). Let \(k, x\) be in \(\Z\). We define the congruence class \(\cls{k}\) of the
          integer \(k\) as the set
          \[
              \begin{aligned}
                  \cls{k} & = \{ x \in \Z \mid x - k \equiv 0 \mod n \}               \\
                          & = \{ x \in \Z \mid \exists a \in \Z: x - k = n \cdot a \}
              \end{aligned}
          \]

          We now define \(\Z / n\Z\) (also \(\Z_n\)) as the set of all congruence classes modulo \(n\).
          Euclidean division implies that this is a finite set of \(n\) elements:
          \[
              \Z_n = \left \{ \cls{0}, \cls{1}, \ldots, \cls{n - 1} \right \}.
          \]

          For all \(a, b \in \Z_n\), we define:
          \[
              \cls{a} \oplus \cls{b} = \cls{a + b}
          \]

    \item[a.] Show that (\(\Z_n, \oplus\)) is a group. Is it Abelian?

          \subsubsection*{Neutral element}

          We have \(\cls{0} \in \Z_n\) such that:
          \[
              \begin{aligned}
                  \cls{a} \oplus \cls{0} = \cls{a + 0} = \cls{a}, & \quad \textrm{and} \\
                  \cls{0} \oplus \cls{a} = \cls{0 + a} = \cls{a}.
              \end{aligned}
          \]

          \subsubsection*{Commutativity}

          For all \(\cls{a}, \cls{b} \in \Z_n\), we have:
          \[
              \begin{aligned}
                  \cls{a} \oplus \cls{b} & = \cls{a + b}             \\
                                         & = \cls{b + a}             \\
                                         & = \cls{b} \oplus \cls{a}.
              \end{aligned}
          \]

          \subsubsection*{Associativity}

          For all \(a, b, c \in \Z_n\), we have:
          \[
              \begin{aligned}
                  (\cls{a} \oplus \cls{b}) \oplus \cls{c} & = \cls{a + b} \oplus \cls{c}               \\
                                                          & = \cls{(a + b) + c}                        \\
                                                          & = \cls{a + (b + c)}                        \\
                                                          & = \cls{a} \oplus \cls{b + c}               \\
                                                          & = \cls{a} \oplus (\cls{b} \oplus \cls{c}).
              \end{aligned}
          \]

          \subsubsection*{Existence of inverse}


          For all \(\cls{a} \in \Z_n\), we require the existence of an element, \(\cls{b} \in \Z_n\), such that:
          \[
              \cls{a} \oplus \cls{b} = \cls{b} \oplus \cls{a} = \cls{0}.
          \]
          We first note that in \(\Z_n\), \(\cls{n} = \cls{0}\), and since \(n - a \in \Z\), its congruence class \(\cls{n - a} \in \Z_n\).

          Supposing then that \(\cls{b} = \cls{n - a}\), we have:
          \[
              \begin{aligned}
                  \cls{a} \oplus \cls{b} & = \cls{a} \oplus \cls{n - a} \\
                                         & = \cls{a + n - a}            \\
                                         & = \cls{n}                    \\
                                         & =       \cls{0}
              \end{aligned}
          \]
          as required, and commutativity gives us \(\cls{b} \oplus \cls{a} = \cls{0}\).

          \subsubsection*{Closure under \(\oplus\)}

          By definition, we have that
          \[
              \cls{a} \oplus \cls{b} = \cls{a + b}
          \]

          Since \(\Z_n\) is the set of congruence classes \(\cls{0}, \cls{1}, \ldots, \cls{n-1}\), and
          every integer has a unique representation modulo \(n\), then given \(a + b \in \Z\), their congruence class
          \(\cls{a + b} \in \Z_n\).

    \item[b.]

          We now define another operation \(\otimes\) for all \(\cls{a}, \cls{b} \in \Z_n\),
          \[
              \cls{a} \otimes \cls{b} = \cls{a \times b},
          \]
          where \(\times\) represents the usual multiplication in \(\Z\).

          We then have the following multiplication table for \(\Z_5 \setminus \{\cls{0}\}\) under \(\otimes\):
          \[
              \begingroup
              \renewcommand{\arraystretch}{1.3}
              \begin{array}{c|ccccc}
                  \otimes & \cls{1} & \cls{2} & \cls{3} & \cls{4} \\
                  \hline
                  \cls{1} & \cls{1} & \cls{2} & \cls{3} & \cls{4} \\
                  \cls{2} & \cls{2} & \cls{4} & \cls{1} & \cls{3} \\
                  \cls{3} & \cls{3} & \cls{1} & \cls{4} & \cls{2} \\
                  \cls{4} & \cls{4} & \cls{3} & \cls{2} & \cls{1} \\
              \end{array}
              \endgroup
          \]

          It follows that \(\Z_5 \setminus \{\cls{0}\}\) is closed under \(\otimes\), with the neutral element
          \(\cls{1}\). From the symmetry about the diagonal, we can immediately conclude that \(\otimes\) commutes.  For
          the inverse, we find the column (resp. row) that yields \(\cls{1}\) for a given row (resp. column), noting
          that \(\cls{1}\) appears in every row (resp. column).  For associativity, we note that for any three
          \(\cls{a}, \cls{b}, \cls{c} \in \Z_n \setminus \{\cls{0}\}\), both \(\cls{a} \otimes (\cls{b} \otimes
          \cls{c})\) and \((\cls{a} \otimes \cls{b}) \otimes \cls{c}\) yield the same result.  Hence, \((\Z_5 \setminus
          \{\cls{0}\}, \otimes)\) forms an Abelian group.

    \item[c.] We find that \((\Z_8 \setminus \{\cls{0}\}, \otimes)\) does not form a group, since \( \cls{2} \otimes
          \cls{4} = \cls{0} \), so closure is not satisfied.

    \item[d.] Bézout's lemma tells us that two integers \(a\) and \(b\) are relatively prime (that is, \(\gcd(a, b) = 1\)) if
          and only if there exist two integers \(u\) and \(v\) such that \(au + bv = 1\).

          Show that \((\Z_n \setminus \{\cls{0}\}, \otimes)\) is a group if and only if \(n \in \N \setminus \{0\}\) is
          prime.

          \begin{proof}[Proof]
              The neutral element is \(\cls{1} \in \Z_n \setminus \{\cls{0}\}\), given that for all \(\cls{a} \in \Z_n
              \setminus \{\cls{0}\}\):
              \[
                  \begin{aligned}
                      \cls{a} \otimes \cls{1} = \cls{a \times 1} = \cls{a} & \quad \textrm{and} \\
                      \cls{1} \otimes \cls{a} = \cls{1 \times a} = \cls{a} & .
                  \end{aligned}
              \]

              For all \(\cls{a}, \cls{b}, \cls{c} \in \Z_n \setminus \{\cls{0}\}\), we have associativity (which follows
              directly from associativity of integer multiplication):
              \[
                  \begin{aligned}
                      (\cls{a} \otimes \cls{b}) \otimes \cls{c}
                      = \cls{a \times b \times c}
                      = \cls{a} \otimes (\cls{b} \otimes \cls{c}).
                  \end{aligned}
              \]

              If \(n\) is composite, then there exist \(\cls{p}, \cls{q} \in \Z_n \setminus \{\cls{0}\}\) such that
              \(\cls{p} \otimes \cls{q} = \cls{0}\); that is, \(\Z_n \setminus \{\cls{0}\}\) contains zero divisors, and
              is not closed under \(\otimes\).  Conversely, if \(n\) is prime then no such \(\cls{p}\) or \(\cls{q}\)
              exist; \(\Z_n \setminus \{\cls{0}\}\) contains no zero divisors and is closed under \(\otimes\).

              Finally, for the existence of an inverse for all \(\cls{a} \in \Z_n \setminus \{\cls{0}\}\), we have from
              Bézout's lemma that \(a\) and \(n\) are relatively prime if and only if there exist two integers \(u\) and
              \(v\) such that \(au + nv = 1\).  Reducing both sides modulo \(n\), we have
              \[
                  au \equiv 1 \mod n,
              \]
              which tells us that \(u\) exists if and only if \(n\) is prime.  Restated:
              \[
                  \left (
                  \forall\, \cls{a} \in \Z_n \setminus \{ \cls{0} \},\quad
                  \exists\, \cls{u} \in \Z_n \setminus \{ \cls{0} \}:\quad
                  \cls{a} \otimes \cls{u} = \cls{1}
                  \right )
                  \iff n \text{ is prime}.
              \]

              Concluding, a neutral element always exists; associativity always holds; closure and the existence of
              inverses hold if and only if \(n\) is prime.  Therefore, \((\Z_n \setminus \{\cls{0}\}, \otimes)\) forms a
              group if and only if \(n\) is prime.

          \end{proof}

    \item[2.3] Consider the set \(\G\) of \(3 \times 3\) matrices, defined as follows
          \[
              \G = \left \{
              \begin{bmatrix}
                  1 & x & z \\
                  0 & 1 & y \\
                  0 & 0 & 1
              \end{bmatrix}
              \in \R^3
              \;\middle|\;
              x, y, z \in \R
              \right \}
          \]
          We define \(\cdot\) as the standard matrix multiplication.  Is \((\G, \cdot)\) a group?  If yes, is it
          Abelian?

          For \(a, b, c, x, y, z \in \R\), let:
          \[
              \mathbf{A} =
              \begin{bmatrix}
                  1 & a & c \\
                  0 & 1 & b \\
                  0 & 0 & 1
              \end{bmatrix}
              \quad
              \mathbf{B} =
              \begin{bmatrix}
                  1 & x & z \\
                  0 & 1 & y \\
                  0 & 0 & 1
              \end{bmatrix}
          \]
          Then we have closure:
          \[
              \mathbf{B} \cdot \mathbf{A} =
              \begin{bmatrix}
                  1 & a + x & c + bx + z \\
                  0 & 1     & b + y      \\
                  0 & 0     & 1
              \end{bmatrix}
              \in \G,
          \]
          and
          \[
              \mathbf{A} \cdot \mathbf{B} =
              \begin{bmatrix}
                  1 & a + x & c + ay + z \\
                  0 & 1     & b + y      \\
                  0 & 0     & 1
              \end{bmatrix}
              \in \G.
          \]

          Associativity follows from associativity of standard matrix multiplication.  Similarly, we have the standard identity
          matrix, \(\mathbf{I} \in \G\).  For \(\mathbf{A}\) defined as above, we have its inverse:
          \[
              \mathbf{A}^{-1} =
              \begin{bmatrix}
                  1 & -a & ab - c \\
                  0 & 1  & -b     \\
                  0 & 0  & 1
              \end{bmatrix}
              \in \G.
          \]
          Verifying, we have
          \[
              \mathbf{A} \cdot \mathbf{A}^{-1} =
              \begin{bmatrix}
                  1 & a - a & ab -c -ab + c \\
                  0 & 1     & b - b         \\
                  0 & 0     & 1
              \end{bmatrix}
              =
              \begin{bmatrix}
                  1 & 0 & 0 \\
                  0 & 1 & 0 \\
                  0 & 0 & 1
              \end{bmatrix}
              = \mathbf{I}
          \]
          and
          \[
              \mathbf{A}^{-1} \cdot \mathbf{A} =
              \begin{bmatrix}
                  1 & a - a & c - ab + ab - c \\
                  0 & 1     & b - b           \\
                  0 & 0     & 1
              \end{bmatrix}
              =
              \begin{bmatrix}
                  1 & 0 & 0 \\
                  0 & 1 & 0 \\
                  0 & 0 & 1
              \end{bmatrix}
              = \mathbf{I}
          \]
          as desired.  We saw above that, in general, multiplication of \(\mathbf{A}, \mathbf{B} \in \G\) does not commute.
          We therefore conclude that \((\G, \cdot)\) is a group, but not Abelian.

\end{enumerate}

\end{document}
