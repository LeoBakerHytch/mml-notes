\documentclass[11pt]{article}

% Input encoding
\usepackage[utf8]{inputenc}

% Geometry
\usepackage{geometry}
\geometry{margin=1in}
\setlength{\parskip}{1em} % Vertical space between paragraphs
\setlength{\parindent}{0pt} % Remove indentation

% Subliminal refinements towards typographical perfection
\usepackage[tracking=true]{microtype}

% Subtitle
\usepackage{titling}

% Good old Computer Modern
\usepackage{amsfonts}
\usepackage{amssymb}
\usepackage{bm}
\usepackage[T1]{fontenc}

% Mathematical typesetting
\usepackage{amsmath} % Base
\usepackage{amsthm} % Theorems
\usepackage{mathtools} % Tools to use with amsmath
\usepackage{xfrac} % Split-level fractions

% Generic symbols
\usepackage{textcomp}
\usepackage{gensymb}

% Plotting
\usepackage{tikz}            % Plots
\usepackage{tikz-3dplot}     % 3D plots
\usetikzlibrary{arrows.meta, positioning}

% Hyperlinks
\usepackage[
    colorlinks=true,
    linkcolor=blue,
    urlcolor=blue,
    citecolor=blue
]{hyperref}

% Custom packages
\usepackage{../common/my-headings}
\usepackage{../common/my-list-control}
\usepackage{../common/my-math-macros}
\usepackage{../common/my-matrix-control}
\usepackage{../common/my-subtitle}
\usepackage{../common/my-theorem-envs}


\usepackage{../common/my-list-control}

\title{Calculus}
\subtitle{Exercises 1: Basic properties of numbers}
\author{}
\date{}

% List spacing
\setlist{itemsep=1em}
\setlist{parsep=0.8em}
% \setlength{\jot}{8pt}

\begin{document}
\maketitle
\begin{enumerate}

    \item[1.] Prove the following:

    \item[(i)] If $ax = a$ for some number $a \neq 0$, then $x = 1$.
          \[
              \begin{alignedat}{2}
                             &  & ax           & = a          \\
                  \iff \quad &  & (a^{-1}) a x & = (a^{-1}) a \\
                  \iff \quad &  & 1 \cdot x    & = 1          \\
                  \iff \quad &  & x            & = 1.
              \end{alignedat}
          \]

    \item[(ii)]
          \[
              \begin{aligned}
                  x^2 - y^2 & = (x - y)(x + y)        \\
                            & = x (x + y) - y (x + y) \\
                            & = x^2 + xy - yx - y^2   \\
                            & = x^2 + xy - xy - y^2   \\
                            & = x^2 - y^2.
              \end{aligned}
          \]

    \item[(iii)] If $x^2 = y^2$, then $x = y$ or $x = -y$.

          \vspace{6pt}
          Taking the square root of both sides:
          \[
              \begin{aligned}
                             & \sqrt{x^2} = \sqrt{y^2} \\
                  \iff \quad & \abs{x} = \abs{y}.
              \end{aligned}
          \]

    \item[(iv)]
          \[
              \begin{aligned}
                  x^3 - y^3 & = (x - y)(x^2 + xy + y^2)               \\
                            & = x(x^2 + xy + y^2) - y(x^2 + xy + y^2) \\
                            & = x^3 + x^2y + xy^2 - (x^2 + xy + y^2)y \\
                            & = x^3 + x^2y + xy^2 - x^2y + xy^2 + y^3 \\
                            & = x^3 + y^3.
              \end{aligned}
          \]

    \item[(v)]
          \[
              \begin{aligned}
                  x^n - y^n & = (x - y)(x^{n - 1} + x^{n - 2} y + \dots + xy^{n - 2} + y^{n - 1})                                \\[6pt]
                            & = x (x^{n - 1} + x^{n - 2}y + \dots + xy^{n - 2} + y^{n - 1})                                      \\
                            & - y (x^{n - 1} + x^{n - 2}y + \dots + xy^{n - 2} + y^{n - 1})                                      \\[6pt]
                            & \begin{alignedat}{3}
                                   & = x^{n} & +  \, &  &  & x^{n - 1}y + x^{n - 2}y^2 + \dots + x^2y^{n - 2} + xy^{n - 1}           \\
                                   &         & - (   &  &  & x^{n - 1}y + x^{n - 2}y^2 + \dots + x^2y^{n - 2} + xy^{n - 1} ) - y^{n} \\
                              \end{alignedat}
                  \\[6pt]
                            & = x^n - y^n.
              \end{aligned}
          \]
          More succinctly:
          \[
              x^n - y^n = (x - y)
              \sum_{i=1}^{n} x^{n - i} y^{i - 1}
          \]

    \item[(vi)] Taking the result that:
          \[
              x^3 - y^3 = (x - y)(x^2 + xy + y^2),
          \]
          substitute $(-y)$ for $y$, which gives us:
          \[
              \begin{aligned}
                             & x^3 - (-y)^3 = (x - (-y))(x^2 + x(-y) + (-y)^2), \\
                  \iff \quad & x^3 + y^3 = (x + y)(x^2 - xy + y^2).
              \end{aligned}
          \]

          Making the same substitution into the result from (v), we have:
          \[
              x^n + y^n = (x + y)
              \sum_{i = 1}^{n} (-1)^{i - 1} x^{n - i} y^{i - 1}
              \quad
              \text{for odd $n$}.
          \]

    \item[2.] In the `proof' that $2 = 1$, since $x = y$ by definition, dividing out the factor $(x
              - y)$ is equivalent to dividing by zero, the result of which is not defined.

    \item[3.]

    \item[(-i)] Prove $a^{-1} b^{-1} = (ab)^{-1}$.

          By the definition of inverses,
          \[
              \begin{alignedat}{2}
                             &  & ab \cdot (ab)^{-1}              & = 1              \\
                  \iff \quad &  & a^{-1} \cdot ab \cdot (ab)^{-1} & = a^{-1}         \\
                  \iff \quad &  & b \cdot (ab)^{-1}               & = a^{-1}         \\
                  \iff \quad &  & b^{-1} \cdot b \cdot (ab)^{-1}  & = b^{-1} a^{-1}  \\
                  \iff \quad &  & (ab)^{-1}                       & = a^{-1} b^{-1}. \\
              \end{alignedat}
          \]

    \item[(i)] Prove:
          \[
              \frac{a}{b} = \frac{ac}{bc}, \quad \text{if $b, c \neq 0$}.
          \]
          \vspace{6pt}
          \[
              \frac{a}{b}
              = a \cdot \frac{1}{b}
              = a \cdot \frac{c}{c} \cdot \frac{1}{b}
              = a \cdot c \cdot \frac{1}{c} \cdot \frac{1}{b}
              = ac \cdot \frac{1}{bc}
              = \frac{ac}{bc}.
          \]

    \item[(ii)] Prove:
          \[
              \frac{a}{b} + \frac{c}{d} = \frac{ad + bc}{bd}, \quad \text{if $b, d \neq 0$}.
          \]
          \vspace{6pt}
          \[
              \begin{aligned}
                  \frac{a}{b} + \frac{c}{d}
                  = \frac{a}{b} \cdot \frac{d}{d} + \frac{b}{b} \cdot \frac{c}{d}
                  = \frac{ad}{bd} + \frac{bc}{bd}
                  = ad \cdot \frac{1}{bd} + bc \cdot \frac{1}{bd}
                  = (ad + bc) \cdot \frac{1}{bd}
                  = \frac{ad + bc}{bd}.
              \end{aligned}
          \]

    \item[(iii)] See (-i).

    \item[(iv)]
          \[
              \frac{a}{b} \cdot \frac{c}{d}
              = a \cdot \frac{1}{b} \cdot c \cdot \frac{1}{d}
              = ac \cdot \frac{1}{d} \cdot \frac{1}{b}
              = ac \cdot \frac{1}{db}
              = \frac{ac}{db}.
          \]

    \item[(v)]
          \[
              \dfrac{a}{b} \, \bigg/ \, \dfrac{c}{d} = \dfrac{ad}{bc}, \quad \text{ if } b,c,d \neq 0.
          \]
          \vspace{6pt}
          \[
              \begin{alignedat}{3}
                  \dfrac{a}{b} \, \bigg/ \, \dfrac{c}{d}
                   & = \dfrac{a}{b} \cdot \left(c \cdot \dfrac{1}{d}\right)^{-1}              & \qquad & \text{(definition of division)}        \\
                   & = a \cdot \dfrac{1}{b} \cdot c^{-1} \cdot \left(\dfrac{1}{d}\right)^{-1} &        & \text{(by (iii) inverse of a product)} \\
                   & = a \cdot \dfrac{1}{b} \cdot \dfrac{1}{c} \cdot d                        &        & \text{(definition of inverses)}        \\
                   & = ad \cdot \dfrac{1}{bc}                                                 &        & \text{(commutativity \& grouping)}     \\
                   & = \dfrac{ad}{bc}.                                                        &        & \text{(definition of division)}
              \end{alignedat}
          \]

    \item[(vi)] If $b, d \neq 0$, then $\dfrac{a}{b} = \dfrac{c}{d}$ if and only if $ad = bc$. Also determine when $\dfrac{a}{b} = \dfrac{b}{a}$.
          \[
              \begin{alignedat}{3}
                             &  & \frac{a}{b}                 & = \frac{c}{d}                  & \qquad &                                                \\
                  \iff \quad &  & a \cdot \frac{1}{b}         & = c \cdot \frac{1}{d}          &        & \text{(definition of division)}                \\
                  \iff \quad &  & a \cdot \frac{1}{b} \cdot b & = c \cdot \frac{1}{d} \cdot b  &        & \text{(multiply by $b$)}                       \\
                  \iff \quad &  & a \cdot 1                   & = b \cdot c \cdot \frac{1}{d}  &        & \text{(multiplicative inverse; commutativity)} \\
                  \iff \quad &  & ad                          & = bc \cdot \frac{1}{d} \cdot d &        & \text{(multiply by $d$)}                       \\[4pt]
                  \iff \quad &  & ad                          & = bc.                          &        & \text{(multiplicative inverse)}
              \end{alignedat}
          \]

          Therefore, we find that
          \[
              \frac{a}{b} = \frac{b}{a} \quad \iff \quad a^2 = b^2,
          \]

          which from (iii) holds when either $a = b$ or $a = -b$.

    \item[4.] Find all numbers for which

    \item[(i)]
          \[
              \setcond{ x \in \R }{ 4 - x < 3 - 2x} = (-\infty, -1).
          \]

    \item[(ii)]
          \[
              \setcond{ x \in \R }{ 5 - x^2 < 8} = \R.
          \]

    \item[(iii)]
          \[
              \setcond{ x \in \R }{ 5 - x^2 < -2 }
              =
              \left( -\infty, -\sqrt{7} \right)
              \cup
              \left( \sqrt{7}, \infty \right).
          \]

    \item[(iv)]
          \[
              \setcond{ x \in \R }{ (x - 1)(x - 3) > 0 } = (-\infty, 1) \cup (3, \infty).
          \]

    \item[(v)]
          \[
              \setcond{ x \in \R }{ x^2 - 2x + 2 > 0 } = \R.
          \]

    \item[(vi)]
          \[
              \setcond{ x \in \R }{ x^2 + x + 1 > 2 } =
              \left(-\infty, \tfrac{-1 - \sqrt{5}}{2}\right)
              \cup
              \left(\tfrac{-1 + \sqrt{5}}{2}, \infty\right).
          \]


    \item[(vii)]
          \[
              \setcond{ x \in \R }{ x^2 - x + 10 > 16 }
              = (-\infty, -2) \cup (3, \infty).
          \]

    \item[(viii)]
          \[
              \setcond{ x \in \R }{ x^2 + x + 1 > 0 } = \R.
          \]

    \item[(ix)]
          \[
              \setcond{ x \in \R }{ (x - \pi)(x + 5)(x - 3) > 0 }
              = (-5, 3) \cup (\pi, \infty).
          \]

    \item[(x)]
          \[
              \setcond{ x \in \R }{ (x - \sqrt[3]{2})(x - \sqrt{2}) > 0 }
              = (-\infty, \sqrt[3]{2}) \cup (\sqrt{2}, \infty).
          \]

    \item[(xi)]
          \[
              \setcond{ x \in \R }{ 2^x < 8 } = (-\infty, 3).
          \]

    \item[(xii)]
          \[
              \setcond{ x \in \R }{ x + 3^x < 4 } = (-\infty, 1).
          \]

    \item[(xiii)]
          \[
              \setcond{ x \in \R }{ \dfrac{1}{x} + \dfrac{1}{1 - x} > 0 } = (0, 1).
          \]

    \item[(xiv)]
          \[
              \setcond{ x \in \R }{ \dfrac{x - 1}{x + 1} > 0 } = (-\infty, -1) \cup (1, \infty).
          \]

    \item[5.] Prove the following:

    \item[(i)] If $a < b$ and $c < d$, then $a + c < b + d$.

          \[
              \begin{aligned}
                  \text{Since \quad}     & a < b \iff (a - b) < 0, \\
                  \text{and \quad}       & c < b \iff (c - d) < 0, \\
                  \text{then \quad}      & (a - b) + (c - d) < 0,  \\
                  \text{therefore \quad} & a + c - d < b,          \\
                  \text{finally \quad}   & a + c < b + d.
              \end{aligned}
          \]

    \item[7.] Prove that if $0 < a < b$, then
          \[
              a < \sqrt{ab} < \frac{a + b}{2} < b.
          \]
          \begin{proof}
              Since $a, b > 0$, from $a < b$ find (multiplying by $a$ resp. $b$ preserves sign):
              \[
                  \begin{aligned}
                      a^2 < ab \implies a < \sqrt{ab}, \\
                      ab < b^2 \implies \sqrt{ab} < b. \\
                  \end{aligned}
              \]
              Thus, the geometric mean, $a < \sqrt{ab} < b$.  For the arithmetic mean,
              consider:
              \[
                  \begin{alignedat}{3}
                      a < b & \iff &  & 2a < a + b & \iff & a < \frac{a + b}{2}, \\
                      a < b & \iff &  & a + b < 2b & \iff & \frac{a + b}{2} < b. \\
                  \end{alignedat}
              \]
              So $a < \tfrac{a + b}{2} < b$.  Finally, to compare the two means:
              \[
                  \begin{alignedat}{3}
                                 &  & \sqrt{ab} & < \frac{a + b}{2} \\
                      \iff \quad &  & 4ab       & < (a + b)^2       \\
                      \iff \quad &  & 0         & < a^2 - 2ab + b^2 \\
                      \iff \quad &  & 0         & < (b - a)^2,      \\
                  \end{alignedat}
              \]
              which holds since $a \neq b$.  Thus, as claimed,
              \[
                  a < \sqrt{ab} < \frac{a + b}{2} < b.
              \]
          \end{proof}

          \begin{proof}[Proof (terse)]
              For $a,b > 0$ with $a < b$:
              \[
                  a^2 < ab \implies a < \sqrt{ab},
                  \qquad
                  ab < b^2 \implies \sqrt{ab} < b,
              \]
              so $a < \sqrt{ab} < b$. Similarly,
              \[
                  2a < a + b < 2b \implies a < \frac{a+b}{2} < b.
              \]
              Finally,
              \[
                  \sqrt{ab} < \frac{a+b}{2}
                  \iff 0 < (b-a)^2,
              \]
              which is true since $a < b$. Hence
              \[
                  a < \sqrt{ab} < \frac{a+b}{2} < b.
              \]
          \end{proof}

          \pagebreak

    \item[9.] Express each of the following with at least one less pair of absolute value
          signs:

    \item[(i)] $\abs{\sqrt{2} + \sqrt{3} - \sqrt{5} + \sqrt{7}} = \sqrt{2} + \sqrt{3} - \sqrt{5} + \sqrt{7}$.

    \item[(ii)] $\abs{(\abs{a + b} - \abs{a} - \abs{b})} = \abs{a} + \abs{b} - \abs{a + b}$.

    \item[(iii)] $\abs{(\abs{a + b} + \abs{c} - \abs{a + b + c})}$ = $\abs{a + b} + \abs{c} - \abs{a + b + c}$.

    \item[(iv)] $\abs{x^2 - 2xy + y^2} = (x-y)^2$.

    \item[(v)] $\abs{(\abs{\sqrt{2} + \sqrt{3}} - \abs{\sqrt{5} - \sqrt{7}})} = \sqrt{2} + \sqrt{3} + \sqrt{5} - \sqrt{7}$.

          \vspace{1em}

    \item[10.] Express each of the following without absolute value signs, treating various cases
          separately when necessary.

    \item[(i)] $
              \abs{a + b} - \abs{b} =
              \begin{cases}
                  a       & b \geq 0,\ a \geq -b, \\
                  -a - 2b & b \geq 0,\ a < -b,    \\
                  -a      & b < 0,\ a < -b,       \\
                  a + 2 b & b < 0,\ a \geq -b.    \\
              \end{cases}
          $

    \item[(ii)] $
              \abs{(\abs{x} - 1)} =
              \begin{cases}
                  -x - 1 & -1 \leq x,     \\
                  x + 1  & -1 < x \leq 0, \\
                  -x + 1 & 0 < x \leq 1,  \\
                  x - 1  & 1 < x.
              \end{cases}
          $

    \item[(iii)] $
              \abs{x} - \abs{x^2} =
              \begin{cases}
                  -x - x^2 & x < 0,    \\
                  x - x^2  & x \geq 0.
              \end{cases}
          $

    \item[(iv)] $
              a - \abs{(a - \abs{a})} =
              \begin{cases}
                  3a & a < 0,    \\
                  a  & a \geq 0.
              \end{cases}
          $

    \item[11.] Find all numbers for which:

    \item[(i)] $
              \abs{x - 3} = 8
              \implies x \in \set{-5, 11}.
          $

    \item[(ii)] $
              \abs{x - 3} < 8
              \implies x \in (-5, 11).
          $

    \item[(iii)] $
              \abs{x + 4} < 2
              \implies x \in (-6, -2).
          $

    \item[(iv)] $
              \abs{x - 1} + \abs{x - 2} > 1
              \implies
              x \in (-\infty, 1) \cup (2, \infty)
          $
          \[
              \abs{x - 1} + \abs{x - 2} - 1 =
              \begin{cases}
                  -2x + 2 & x \leq 1,     \\
                  -1      & 1 < x \leq 2, \\
                  2x - 4  & 2 < x.
              \end{cases}
          \]

    \item[(v)] $
              \abs{x - 1} + \abs{x - 2} < 2
              \implies x \in
              \left(-\infty, \tfrac{1}{2} \right)
              \cup
              \left(\tfrac{5}{2}, \infty \right)
          $
          \[
              \abs{x - 1} + \abs{x - 2} - 2 =
              \begin{cases}
                  -2x + 1 & x \leq 1,     \\
                  -1      & 1 < x \leq 2, \\
                  2x - 5  & 2 < x.
              \end{cases}
          \]

    \item[(vi)] $
              \abs{x - 1} + \abs{x + 1} < 1
              \implies x \in \text{\O}.
          $
          \[
              \abs{x - 1} + \abs{x + 1} - 1 =
              \begin{cases}
                  -2x - 1 & x < -1,           \\
                  1       & -1 \leq x \leq 1, \\
                  2x -1   & x > 1.            \\
              \end{cases}
          \]

    \item[(vii)] $
              \abs{x - 1} \cdot \abs{x + 1} = 0
              \implies x \in \set{-1, 1}.
          $


    \item[(viii)] $
              \abs{x - 1} \cdot \abs{x + 2} = 3
              \implies x \in \set{
                  \frac{-1 \pm \sqrt{21}}{2}
              }.
          $

    \item[($\star$)] Given $a < b$ and $k \geq 0$, solve for x:
          \[
              \abs{x - a} \cdot \abs{x - b} = k.
          \]

          This divides the number line into three segments:
          \[
              (-\infty, a], [a, b], [b, \infty).
          \]
          Analysing piecewise:
          \[
              \abs{x - a} \cdot \abs{x - b} - k =
              \begin{cases}
                  -x^2 + (a - b)x - (ab + k) & a \leq x \leq b,  \\
                  x^2 - (a + b)x + ab - k    & \text{otherwise}. \\
              \end{cases}
          \]


          Evaluate at the midpoint of the interior interval, $\tfrac{1}{2}(a + b)$:
          \[
              \begin{aligned}
                  \mathcal{D} = & \abs{\tfrac{1}{2}(a + b) - a} \cdot \abs{\tfrac{1}{2}(a + b) - b} \\
                  =             & \abs{\tfrac{1}{2}(b - a)} \cdot \abs{\tfrac{1}{2}(a - b)}         \\
                  =             & \tfrac{1}{4} (b - a)^2.
              \end{aligned}
          \]
          So, for $k > \mathcal{D}$, the interior interval has no solutions; for $k = \mathcal{D}$,
          one repeated solution; for $0 \leq k < \mathcal{D}$, two distinct solutions; and for $k <
              0$ there are no solutions at all, since the entire function lies above the line $y = k$
          (if we plot the function).

          Solving for $a \leq x \leq b$:
          \[
              \tfrac{1}{2} \left( a - b \pm \sqrt{a^2 + b^2 - 6ab - 4k} \right).
          \]
          Solving for $x < a$ and $b < x$:
          \[
              \tfrac{1}{2} \left( a + b \pm \sqrt{a^2 + b^2 - 2ab - 4k} \right).
          \]

    \item[12.] Prove the following:

    \item[(i)] $\abs{xy} = \abs{x} \cdot \abs{y}$.
          \begin{proof}
              For any real number, $t$, we have $\abs{t}^2 = t^2$.  Consider,
              \[
                  \abs{xy}^2 = x^2 y^2 = \abs{x}^2 \cdot \abs{y}^2.
              \]
              Since both sides are non-negative, taking square roots gives:
              \[
                  \abs{xy} = \abs{x} \cdot \abs{y}.
              \]
          \end{proof}

    \item[(ii)] $\left\lvert \dfrac{1}{x} \right\rvert = \dfrac{1}{\abs{x}}$, if $x \neq 0$.
          \begin{proof}
              As before, for any real number, $t$, we have $\abs{t}^2 = t^2$. Therefore,
              \[
                  \left\lvert \dfrac{1}{x} \right\rvert^2
                  = \dfrac{1}{x^2}
                  = \dfrac{1}{\abs{x}^2}.
              \]
              Once again, both sides are non-negative; taking square roots gives the desired result:
              \[
                  \left\lvert \dfrac{1}{x} \right\rvert
                  = \dfrac{1}{\abs{x}}.
              \]
          \end{proof}

    \item[(iii)] $\dfrac{\abs{x}}{\abs{y}} = \left\lvert \dfrac{x}{y} \right\rvert$, if $y \neq 0$.
          \begin{proof}
              \[
                  \left\lvert \dfrac{x}{y} \right\rvert^2 =
                  \dfrac{x^2}{y^2} =
                  \dfrac{\abs{x}^2}{\abs{y}^2}.
              \]

              Since both sides are non-negative, taking square roots gives:
              \[
                  \left\lvert \dfrac{x}{y} \right\rvert = \dfrac{\abs{x}}{\abs{y}}.
              \]
          \end{proof}

          \pagebreak

    \item[(iv)] $\abs{x - y} \leq \abs{x} + \abs{y}$.
          \begin{proof}
              By definition, $\pm x \leq \abs{x}$ and $\pm y \leq \abs{y}$.  Then, since addition
              preserves inequality order:
              \[
                  \begin{aligned}
                                       &  & x - y             & \leq \abs{x} + \abs{y}, \\
                      \text{and} \quad &  & \mathord{-}(x -y) & \leq \abs{x} + \abs{y},
                  \end{aligned}
              \]
              which bounds $x - y$ from above and below:
              \[
                  -(\abs{x} + \abs{y}) \leq x - y \leq \abs{x} + \abs{y}.
              \]
              Since $\abs{a} \leq b$ is equivalent to $-b \leq a \leq b$, we have:
              \[
                  \abs{x - y} \leq \abs{x} + \abs{y}.
              \]
          \end{proof}

    \item[(v)] $\abs{x} - \abs{y} \leq \abs{x - y}$.
          \begin{proof}
              From (iv), $\abs{a - b} \leq \abs{a} + \abs{b}$, or equivalently $\abs{a - b} -
                  \abs{b} \leq \abs{a}$.  Let $x = a - b$ and let $y = b$.  Then $\abs{x} - \abs{y} \leq
                  \abs{x - y}$, as desired.
          \end{proof}

    \item[(vi)] $\big\lvert\, \abs{x} - \abs{y} \,\big\rvert \leq \abs{x - y}$.
          \begin{proof}
              From (v), $\abs{x} - \abs{y} \leq \abs{x - y}$.  Multiplying by $-1$, reversing the
              direction of the inequality, permuting $x$ and $y$, and noting that $\abs{y - x} =
                  \abs{x - y}$, we have
              \[
                  -\abs{x - y} \leq \abs{x} - \abs{y}.
              \]
              As before, since $\abs{a} \leq b$ is equivalent to $-b \leq a \leq b$, and $\abs{x} -
                  \abs{y}$ is bounded from above and below, we find
              \[
                  \big\lvert\, \abs{x} - \abs{y} \,\big\rvert \leq \abs{x - y}.
              \]
          \end{proof}

    \item[(vii)] $\abs{x + y + z} \leq \abs{x} + \abs{y} + \abs{z}$.
          \begin{proof}
              Since $\pm a \leq \abs{a}$,
              \[
                  \begin{aligned}
                       &
                       &
                      x + y + z
                       &
                      \; \leq \; \abs{x} + y + z
                      \; \leq \; \abs{x} + \abs{y} + z
                      \; \leq \; \abs{x} + \abs{y} + \abs{z}, \\
                      \text{and} \quad
                       &
                       &
                      -x - y - z
                       &
                      \; \leq \; \abs{x} - y - z
                      \; \leq \; \abs{x} + \abs{y} - z
                      \; \leq \; \abs{x} + \abs{y} + \abs{z}.
                  \end{aligned}
              \]
              So $x + y + z$ is bounded from above and below, or equivalently
              \[
                  \abs{x + y + z} \leq \abs{x} + \abs{y} + \abs{z}.
              \]
          \end{proof}

          If and only if $x, y, z \leq 0$, or $x, y, z \geq 0$, then $\abs{x + y + z} = \abs{x} +
              \abs{y} + \abs{z}$.
          \begin{proof}
              For non-negative $x, y, z$, $\abs{x} = x$, $\abs{y} = y$, $\abs{z} = z$, so
              \[
                  \abs{x} + \abs{y} + \abs{z} = x + y + z = \abs{x + y + z}.
              \]
              For non-positive $x, y, z$, $\abs{x} = -x$, $\abs{y} = -y$, $\abs{z} = -z$, so
              \[
                  \abs{x} + \abs{y} + \abs{z} = -(x + y + z) = \abs{x + y + z}.
              \]
              For only if, consider non-negative $x, y$ but negative $z$,
              \[
                  \abs{x} + \abs{y} + \abs{z} = x + y - z < \abs{x + y + z}.
              \]
              The same inequality holds for any other permutation where the signs of $x, y, z$ are
              not all the same.
          \end{proof}

    \item[($\star$)] $\abs{x_1 + \dots + x_n} \leq \abs{x_1} + \dots + \abs{x_n}$.

          \pagebreak

    \item[13.] The maximum of two numbers $x$ and $y$ is denoted by $\max(x, y)$.  The minimum of
          $x$ and $y$ is denoted by $\min(x, y)$.

          $\max(x, y) = \frac{1}{2} \left( x + y + \abs{y - x} \right)$.
          \begin{proof}
              If $x > y$, then
              \[
                  \begin{aligned}
                      \max(x, y) & = x                                  \\
                                 & = \tfrac{1}{2} (x + y + \abs{y - x}) \\
                                 & = \tfrac{1}{2} (x + y + x - y)       \\
                                 & = x.
                  \end{aligned}
              \]
              Since $\abs{x - y} = \abs{y - x}$, the corresponding result holds for the second
              argument.  In the case $x = y$, the equality above holds.
          \end{proof}

          $\min(x, y) = \frac{1}{2} \left( x + y - \abs{y - x} \right)$.
          \begin{proof}
              If $x < y$, then
              \[
                  \begin{aligned}
                      \min(x, y) & = x                                  \\
                                 & = \tfrac{1}{2} (x + y - \abs{y - x}) \\
                                 & = \tfrac{1}{2} (x + y - y + x)       \\
                                 & = x.
                  \end{aligned}
              \]
              Since $\abs{x - y} = \abs{y - x}$, the corresponding result again holds for the second
              argument.  Likewise, in the case $x = y$, the equality above holds.
          \end{proof}

          Derive a formula for $\max(x, y, z)$ and $\min(x, y, z)$.

          \[
              \begin{aligned}
                  \max(x, y, z)
                   &
                  = \max(x, \max(y, z))
                  \\ &
                  = \tfrac{1}{2} ( x + \max(y, z) + \abs{ \max(y, z) - x } )
                  \\ &
                  = \tfrac{1}{2} \left(
                  x
                  + \tfrac{1}{2}(y + z + \abs{z - y})
                  + \big\lvert
                  \tfrac{1}{2}(y + z + \abs{z - y})
                  - x
                  \big\rvert \right)
                  \\ &
                  = \tfrac{1}{4} \left(
                  2x + y + z + \abs{z - y}
                  +
                  \big\lvert
                  y + z + \abs{z - y} - 2x
                  \big\rvert \right)
              \end{aligned}
          \]

          \pagebreak

    \item[14.]

    \item[(a)] Prove that $\abs{a} = \abs{-a}$. (The trick is not to become confused by too many
          cases.  First prove the statement for $a \geq 0$.  Why is it then obvious for $a \leq 0$)?

    \item[(b)] Prove that $-b \leq a \leq b$ if and only if $\abs{a} \leq b$.  In particular it follows that $-\abs{a} \leq a \leq \abs{a}$.

    \item[(c)] Use this fact to give a new proof that $\abs{a + b} \leq \abs{a} + \abs{b}$.

          \pagebreak

    \item[*15.] Prove that if $x$ and $y$ are not both $0$, then
          \[
              \begin{aligned}
                  x^2 + xy + y^2 > 0, \\
                  x^4 + x^3y + x^2y^2 + xy^3 + y^4 > 0.
              \end{aligned}
          \]

          \pagebreak

    \item[*16.]

    \item[(a)] Show that
          \[
              \begin{aligned}
                  (x + y)^2 & = x^2 + y^2 \quad \text{only when $x = 0$ or $y = 0$,}           \\
                  (x + y)^3 & = x^3 + y^3 \quad \text{only when $x = 0$, $y = 0$ or $x = -y.$} \\
              \end{aligned}
          \]

    \item[(b)] Using the fact that
          \[
              x^2 + 2xy + y^2 = (x + y)^2 \geq 0,
          \]
          show that $4x^2 + 6xy + 4y^2 > 0$ unless $x$ and $y$ are both $0$.

    \item[(c)] Use part (b) to find out when $(x + y)^4 = x^4 + y^4$.

    \item[(d)] Find out when $(x + y)^5 = x^5 + y^5$.

          Hint: From the assumption $(x + y)^5 = x^5 + y^5$ you should be able to derive the equation $x^3
              + 2x^2y + 2xy^2 +y^3 = 0$, if $xy \ne 0$.  This implies that $(x + y)^3 = x^2y + xy^2 = xy(x +
              y)$.

          You should now be able to make a good guess as to when $(x + y)^n = x^n + y^n$.

\end{enumerate}
\end{document}